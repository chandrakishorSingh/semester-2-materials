

\documentclass{article}
\usepackage[margin=1in]{geometry} 
\usepackage{amsmath,amsthm,amssymb,amsfonts, fancyhdr, color, comment, graphicx, environ}
\usepackage{xcolor}
\usepackage{mdframed}
\usepackage[shortlabels]{enumitem}
\usepackage{indentfirst}
\usepackage{hyperref}
\renewcommand{\footrulewidth}{0.8pt}
\hypersetup{
    colorlinks=true,
    linkcolor=blue,
    filecolor=magenta,      
    urlcolor=blue,
}


\pagestyle{fancy}



\newenvironment{problem}[2][Problem]
    { \begin{mdframed}[backgroundcolor=gray!20] \textbf{#1 #2} \\}
    {  \end{mdframed}}


\newenvironment{solution}{\textbf{Solution}}


\lhead{MIT2021117}
\rhead{} 
\chead{\textbf{Assignment 2}}
\lfoot{}
\rfoot{}


\begin{document}
    \begin{problem}{1}
        If $A = U \Sigma V^{T}$ is the SVD of a square invertible matrix $A$, what is the SVD of $A^{-1}$ ?
    \end{problem}
    
    \begin{solution}
        The SVD of a $m x n$ matrix $A$ is given by $A = U W V^{T}$ where,
        
        $U$: $m$ x $n$ orthonormal eigenvectors of $AA^{T}$
        
        $V^{T}$: $n$ x $n$ orthonormal eigenvectors of $A^{T}A$
        
        $W$: $n$ x $n$ diagonal matrix having diagonal elements as the square roots of the eigenvalues of $A^{T}A$.
        
        \hfill
        
        Since, $U$ \& $V$ are orthogonal, $U^{T} = U^{-1}$ and $V^{T} = V^{-1}$
        
        \hfill
        
        A diagonal matrix $D$ with non-zero entries will have an inverse as 
        \[
            D^{-1} = 
            \begin{bmatrix}
                1/d_{1} & 0 & 0 & \dots & 0 \\
                0 & 1/d_{2} & 0 & \dots & 0 \\
                \vdots & \vdots & \vdots & \ddots & \vdots \\
                0 & 0 & 0 & \dots & 1/d_{n} \\
            \end{bmatrix}
        \]
        
        where $d_{1}, d_{1}, \dots, d_{n}$ are elements of $D$. Hence, the diagonal matrix $W$ can also be inverted in the same way.
        
        \hfill
        
        Hence, $A^{-1} = (UWV^{T})^{-1} = (V^{T})^{-1} W^{-1} U^{-1} = VW^{-1}U^{T}$.
    
        
    \end{solution}
    
    \begin{problem}{2}
        If $A = U \Sigma V^{T}$ is the SVD of a square invertible matrix $A$, what is the SVD of $A^{-1}$ ?
    \end{problem}
    
    \begin{solution}
    The Eigendecomposition of a matrix is as follows.
    
    \hfill
    
    The matrix U is formed by placing the eigenvector of A as its column. The $\Lambda$ is a diagonal matrix where diagonal elements are eigenvalues.
    
    \hfill

    So, $AU = U \Lambda$. Hence, $A = U \Lambda U^{-1}$
    
    \hfill
    
    In positive semi-definite matrix, the eigenvalues are greater than or equal to zero. Also, any two eigenvector are orthogonal.
    
    \hfill
    
    Hence, U is a orthogonal matrix and if we normalize its eigenvetor then it becomes orthonormal.
    
    \hfill
    
    So, we can write $A = U \Lambda U^{-1}$.
    
    \hfill
    
    The eigendecomposition of A satisfies all the requirement of having SVD.

    \end{solution}
    
    \begin{problem}{3}
        If A is not positive semidefinite, how can we modify signs in A = $U \Lambda U^T$
    \end{problem}
    
    \begin{solution}
        If A is not positive semi-definite then there there exists  eigenvalues which are not positive. In such case, we can take absolute value of eigen vectors to form the diagonal matrix $\Lambda$.
    \end{solution}
    
    \begin{problem}{4}
        Show that for a square A = $USU^T$ one has $AA^T$ = $US^2U^T$
and $A^TA$ =$V^TS^2V$ , and that these are eigen decompositions.
    \end{problem}
    
    \begin{solution}
        Given, A = USU\textsuperscript{T}\newline
	\hspace*{21mm}So,\hspace*{5mm} A\textsuperscript{T} = (USU\textsuperscript{T})\textsuperscript{T}\newline
	    \hspace*{23mm}$\implies$ A\textsuperscript{T} = (U\textsuperscript{T})\textsuperscript{T}S\textsuperscript{T}U\textsuperscript{T}\newline
	    \hspace*{23mm}$\implies$ A\textsuperscript{T} = US\textsuperscript{T}U\textsuperscript{T}\newline\newline\newline
	   So,\hspace*{5mm} A\textsuperscript{T}A = USU\textsuperscript{T}US\textsuperscript{T}U\textsuperscript{T}\newline
	   \hspace*{2mm}$\implies$ A\textsuperscript{T}A = USS\textsuperscript{T}U\textsuperscript{T}\newline
	   	   \hspace*{2mm}$\implies$ A\textsuperscript{T}A = US\textsuperscript{2}U\textsuperscript{T}\newline\newline
	    So,\hspace*{5mm} AA\textsuperscript{T} = USU\textsuperscript{T}US\textsuperscript{T}U\textsuperscript{T}\newline
	   \hspace*{2mm}$\implies$ AA\textsuperscript{T} = USS\textsuperscript{T}U\textsuperscript{T}\newline
	   	   \hspace*{2mm}$\implies$ AA\textsuperscript{T} = US\textsuperscript{2}U\textsuperscript{T}\newline\newlne
    \end{solution}


\end{document}
