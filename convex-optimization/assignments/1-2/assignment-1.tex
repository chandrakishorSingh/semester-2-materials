

\documentclass{article}
\usepackage[margin=1in]{geometry} 
\usepackage{amsmath,amsthm,amssymb,amsfonts, fancyhdr, color, comment, graphicx, environ}
\usepackage{xcolor}
\usepackage{mdframed}
\usepackage[shortlabels]{enumitem}
\usepackage{indentfirst}
\usepackage{hyperref}
\renewcommand{\footrulewidth}{0.8pt}
\hypersetup{
    colorlinks=true,
    linkcolor=blue,
    filecolor=magenta,      
    urlcolor=blue,
}


\pagestyle{fancy}



\newenvironment{problem}[2][Problem]
    { \begin{mdframed}[backgroundcolor=gray!20] \textbf{#1 #2} \\}
    {  \end{mdframed}}


\newenvironment{solution}{\textbf{Solution}}


\lhead{MIT2021117}
\rhead{} 
\chead{\textbf{Assignment 1}}
\lfoot{}
\rfoot{}


\begin{document}
    \begin{problem}{1}
        What is dual norm. Derive the dual norm of the $L_{1}$, $L_{2}$ and $L_{\infty}$.
    \end{problem}
    
    \begin{solution}
        Let $\Vert w \Vert$  be  a  generic  norm  of  vector w.
        
        \hfill
        
        The dual norm is defined as follows.
        
        \hfill
        
        $\Vert x \Vert$  $\star = max\langle$ w, x  $\rangle$ such that  $\Vert$ w  $\Vert$  $\leq$ 1.
        
        \hfill
        
        Hence, we get the following result.
        
        \hfill
        
        $\langle w, z  \rangle  \leq  \Vert w  \Vert   \Vert z  \Vert$  $\star$
        
        \hfill
        
        \textbf{Dual norm of $L_{1}$ and $L_{\infty}$}
        
        \hfill
 
 
 

    
        Let $||z|| = \Sigma |z_{i}| = ||Z||_{1} (l_{1} norm)$.
        
        \hfill
        
        maximize $\Sigma z_{i}y_{i}$ for $\Sigma |z_{i}| \leq 1$.
        
        \hfill
        
        $= max |y_{i}| = ||y||_{\infty}$.
        
        \hfill
        
        Hence, the dual norm of $L_{1}$ is $L_{\infty}$. Since we know that the dual norm of the dual norm of is the original norm. Hence, the dual norm of $L_{\infty}$ is $L_{1}$.
        
        \hfill
        
        \textbf{Dual norm of $L_{2}$}
        
        We can see from the below equation that dual norm of $L_{2}$ is the $L_{2}$ itself.  
        
        \begin{equation}
            \begin{aligned}
            \max_{||z||_{2} \leq 1} \quad & 
            z^{T}y \leq ||z||_{2} ||y||_{2} \leq ||y||_{2}\\
            \end{aligned}
        \end{equation}
        
        \hfill
        
        The equality is obtained when,
        
        \[ z = 
            \begin{cases} 
                ||y||_{2}^{-1}.y , & y \neq 0 \\
                0, & y = 0 \\
            \end{cases}
        \]
        
    \end{solution}
    
    \newpage
    
    \begin{problem}{2}
        Gram-Schmidt Let $(1,-1,1,1), (1,0,1,0), (0,1,0,1)$ be a linearly independent set in $R^4$. Find an orthonormal set $v1, v2, v3$ st L( (1, −1,1,1), (1,0,1,0), (0,1,0,1)) = L(v1, v2, v3 )
    \end{problem}
    
    \begin{solution}
        step-1 \\
   $\mathbf{\vec{u_{1}}} = \mathbf{\vec{v_{1}}} = \left[\begin{array}{c}1\\-1\\1\\1\end{array}\right]$
   $\mathbf{\vec{e_{1}}} = \frac{\mathbf{\vec{u_{1}}}}{\mathbf{\left\lvert\vec{u_{1}}\right\rvert}} = \left[\begin{array}{c}\frac{1}{2}\\- \frac{1}{2}\\\frac{1}{2}\\\frac{1}{2}\end{array}\right]$  \\
   step-2 \\
   $\mathbf{\vec{u_{2}}} = \mathbf{\vec{v_{2}}} - \text{proj}_{\mathbf{\vec{u_{1}}}}\left(\mathbf{\vec{v_{2}}}\right) = \left[\begin{array}{c}\frac{1}{2}\\\frac{1}{2}\\\frac{1}{2}\\- \frac{1}{2}\end{array}\right]$
   $\mathbf{\vec{e_{2}}} = \frac{\mathbf{\vec{u_{2}}}}{\mathbf{\left\lvert\vec{u_{2}}\right\rvert}} = \left[\begin{array}{c}\frac{1}{2}\\\frac{1}{2}\\\frac{1}{2}\\- \frac{1}{2}\end{array}\right]$ \\
   step-3 \\
   $\mathbf{\vec{u_{3}}} = \mathbf{\vec{v_{3}}} - \text{proj}_{\mathbf{\vec{u_{1}}}}\left(\mathbf{\vec{v_{3}}}\right) - \text{proj}_{\mathbf{\vec{u_{2}}}}\left(\mathbf{\vec{v_{3}}}\right) = \left[\begin{array}{c}0\\1\\0\\1\end{array}\right]$ 
   $\mathbf{\vec{e_{3}}} = \frac{\mathbf{\vec{u_{3}}}}{\mathbf{\left\lvert\vec{u_{3}}\right\rvert}} = \left[\begin{array}{c}0\\\frac{\sqrt{2}}{2}\\0\\\frac{\sqrt{2}}{2}\end{array}\right]$ \\
   answer: \\
   $\left\{\left[\begin{array}{c}\frac{1}{2}\\- \frac{1}{2}\\\frac{1}{2}\\\frac{1}{2}\end{array}\right], \left[\begin{array}{c}\frac{1}{2}\\\frac{1}{2}\\\frac{1}{2}\\- \frac{1}{2}\end{array}\right], \left[\begin{array}{c}0\\\frac{\sqrt{2}}{2}\\0\\\frac{\sqrt{2}}{2}\end{array}\right]\right\}\approx \left\{\left[\begin{array}{c}0.5\\-0.5\\0.5\\0.5\end{array}\right], \left[\begin{array}{c}0.5\\0.5\\0.5\\-0.5\end{array}\right], \left[\begin{array}{c}0\\0.707106781186548\\0\\0.707106781186548\end{array}\right]\right\}.$
    \end{solution}
    
    \newpage
    

    \begin{problem}{3}
    Prove that every real, symmetric matrix $X$ has the decomposition $X = Q \Lambda Q^{T}$. $Q$ is an orthogonal matrix and $\Lambda$ is a diagonal matrix with eigenvalues as elements.
    \end{problem}
    
    \begin{solution}
    
    Let $n$ be the order of real symmetric matrix $A$.
    
    % \hfill
    
    And $x_{1}, x_{2}, \hdots, x_{n}$ be the eigenvector of A with corresponding eigenvalues $\lambda_{1}, \lambda_{2}, \hdots, \lambda_{n}$.
    
    \hfill
    
    Let $\Lambda = 
    \begin{bmatrix}
        \lambda_{1} & 0 & 0 & \dots & 0 \\
        0 & \lambda_{2} & 0 & \dots & 0 \\
        \vdots & \vdots & \vdots & \ddots & \vdots \\
        0 & 0 & 0 & \dots & \lambda_{n} \\
    \end{bmatrix}
    $ be the eigenvalue matrix of A. And $Q = [x_{1}, x_{2}, ..., x_{n}]$ be the eigenvector matrix of A.
    
    \hfill
    
    Then, we have,
    
    \hfill
    
    \[
        AQ = A[x_{1}, x_{2}, ..., x_{n}]
    \]
    
    \[
        = [Ax_{1}, Ax_{2}, ..., Ax_{n}]
    \]
    
    \[
        = [\lambda_{1}x_{1}, \lambda_{2}x_{2}, ..., \lambda_{n}x_{n}]
    \]
    
    \[
        = Q\Lambda
    \]
    
    Hence, $AQ = Q\Lambda$.
    
    \hfill
    
    Or, $A = Q \Lambda Q^{-1}$
    
    If we can prove that $Q$ is orthogonal then we would get our desired result of $A = Q \Lambda Q^{T}$ as for any orthogonal matrix, $Q^{-1} = Q^{T}$
    
    \hfill
    
    We know that for any real matrix A and any vectors x and y, we have,
    
    \[
        \langle Ax, y \rangle = \langle x, A^{T}y \rangle
    \]
    
    Let $x$, $y$ be the eigenvector of $A$ corresponding to distinct eigenvalue $\lambda_{1}$ and $\lambda_{2}$. We already know that $A$ is a real, symmetric matrix. Then,
    
    \[
        \lambda_{1}\langle x, y \rangle
        = \langle \lambda_{1}x, y \rangle
        = \langle Ax, y \rangle
        = \langle x, A^{T}y \rangle
        = \langle x, Ay \rangle
        = \langle x, \lambda_{2}y \rangle
        = \lambda_{2}\langle x, y \rangle
    \]
    
    \hfill
    
    Therefore, $(\lambda_{1} - \lambda_{2}) \langle x, y \rangle = 0$. Since $(\lambda_{1} - \lambda_{2}) \neq 0$ as they are distinct eigenvalues. Therefore, $x$ and $y$ are orthogonal to each other.
    
    \hfill
    
    Similarly, we can prove that any two eigenvector of $A$ are orthogonal to each other. Hence, the matrix $Q$ is orthogonal.
    
    \hfill
    
    Hence, $A = Q \Lambda Q^{T}$
    
    \end{solution}
    
    \newpage
    
    \begin{problem}{4}
        Can an orthogonal matrix have an entry $U_{ij} > 1 $? Why?
    \end{problem}
    
    \begin{solution}
        Yes. Since, all diagonal matrix are orthogonal and a diagonal matrix can have a diagonal element greater than 1. Hence, it is possible for an orthogonal matrix to have an element greater than 1.
    \end{solution}
    
    % \newpage
    
    \begin{problem}{5}
        When is a diagonal matrix orthogonal?
    \end{problem}
    
    \begin{solution}
        Every diagonal matrix has the property that it is orthogonal. Hence, every diagonal matrix is already orthogonal.
    \end{solution}
    
    % \newpage
    
    
    \begin{problem}{6}
        When is an upper triangular matrix orthogonal ?
    \end{problem}
    
    \begin{solution}
        Let $A$ be an upper triangular matrix which is also orthogonal. Then, as per orthogonality, $A^{-1} = A^{T}$.
        
        \hfill
        
        Also, note that the inverse of an upper triangular matrix is also an upper triangular matrix. Hence, $A^{T}$ is both upper triangular and lower triangular matrix, i.e., a diagonal matrix. It implies that $A$ is also a diagonal matrix.
        
        \hfill
        
        Hence, an upper triangular matrix will be orthogonal when it is diagonal.
    \end{solution}
    
    % \newpage
    
    \begin{problem}{7}
        Is the inverse of an orthogonal matrix orthogonal ?
    \end{problem}
    
    \begin{solution}
        We know that, $A^T = A^{-1}$ 
        
        \hfill
        
        Taking inverse, $(A^{-1})^{-1} = A$
        
        \hfill
        
        Taking transpose, $(A^T)^T = A$
        
        \hfill
        
        Hence, $(A^{-1})^{-1} = A = (A^T)^T = (A^{-1})^T$
    \end{solution}
    
    % \newpage
    
    \begin{problem}{8}
        What are eigenvalues and eigenvectors of a diagonal matrix ?
    \end{problem}
    
    \begin{solution}
        The eigenvalues of a diagonal matrix are present as the elements of that diagonal matrix. And its eigenvector form a canonical basis for space $K^n$ 
    \end{solution}
    
    % \newpage
    
    \begin{problem}{9}
        Can you find vectors that attain each of the equality cases in $\lambda_{min} \leq \frac{x^{T}Ax}{||x||^{2}} \leq \lambda_{max}$ ?
    \end{problem}
    
    \begin{solution}
        The given inequality is known as Rayleigh Inequality. Here, $A$ is any symmetric matrix of order $n$ and $x$ is any vector of $\mathbb{R}^{n}$.
        
        \hfill
        
        One vector for which equality can be attain in both the cases is when $x = \begin{bmatrix}
            1 & 0 & 0
        \end{bmatrix}$ and A is an identity matrix of order 3.
        
        \hfill
        
        Then $\lambda_{min}$ and $\lambda_{max}$ both are 1 and $||x||^2 = 1.1 + 0.0 + 0.0 = 1$ and $x^TAx = 1$.
        
        \hfill
        
        There are many other vectors as well. Any column vector of an identity matrix will also satisfy this property.
        
        
    \end{solution}
    


\end{document}
